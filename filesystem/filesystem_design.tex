\documentclass[a4paper, 12pt]{article}
\usepackage[T2A]{fontenc}
\usepackage[english]{babel}
\usepackage[pdftex, unicode]{hyperref}

\title{Filesystem: Design Document}
\date{June 2, 2011}
\author{Erik Q. Steggall}

\begin{document}
\maketitle

\section{Purpose}
The purpose of this project is to impliment a signature bit onto files. Implimenting this requires an intimate knowledge of the Minix filesystem.\\

\section{Design}
In order to impliment the signature I had to write in functions to compute and write the signature. I modified read.c so that it has the ability to read and write signatures.\\

\section{Design details/psudo code}

{\bf read.c}


{\bf rw\_chunk()}\\
\begin{verbatim}
  rip is the pointer to inode
  bp is the pointer to block
  sbp is the pointer to signature block
  signature is the block signature 
  SIG\_ZONE is the signature zone
  block is the current block
  if(sticky bit is set and file inode){
    scale is the amount to shift by
    if(The zone has been already been created){
      Allocate a zone for the current inode at SIG\_ZONE;
      Get scale to shift by;
      Get a block for sbp to point to;
      zero out sbp
    }else{
      shift block by scale;
      Get a block for sbp to point to;  
    }
    if(READING){
      Signature gets output of compute signature;
      print signature;
    }else{
      Signature gets output of compute signature;
      write signature;
    } 
  } 
}      
\end{verbatim}

{\bf compute\_signature}\\
\begin{verbatim}
  signature is the block signature
  itor = iterator
  for(itor to blocksize, itor+4){
    signature XOR block data;
  }
  return signature
\end{verbatim}

\section{Testing}
My code works for putting in signatures: It writes a new signature every write, and reads the same signature as long a there isn't a write in between. I wasn't able to get the file to lock the user out, but I would imagine adding a signature field to the struct buf, and checking the signature against that.\\
To test, run fsSetup.sh, shutdown, boot up, test out file.\\

\section{Conclusion}
I learned a lot about how the Minix file system works. This was the first project that I felt comfortable working with the Minix source code. For instance, I completely screwed my VM so I had to reinstall. At the beginning of the quarter I would have had no clue of how to get my networking going. This time, I was able to find some documentation online and modified a file in /usr/etc/rc and it worked! I have a lot more repect for how much work and operating system does behind the scenes to allow us to use even the most basic functions of modern computers. I now feel like I could start tinkering with the Linix OS which is what I had hoped to get out of this class.\\ 

      
      
      
    




\end{document}
