\documentclass[a4paper, 12pt]{article}
\usepackage[T2A]{fontenc}
\usepackage[english]{babel}
\usepackage[pdftex, unicode]{hyperref}

\title{Memory: Design Document}
\date{May, 18, 2011}
\author{Erik Q. Steggall}

\begin{document}
\maketitle

\section{Purpose}
The purpose of this project is to write alternate algorithms for the operating system's memory allocation. We are to write next fit, random fit, best fit and worst fit.\\

\section{Design}
I wrote a system call that can change the type of algorithm being used by the operating system's memory allocator. The system call will change the algorithm if it is given a number between one and four. Otherwise it runs the already implimented first fit algorithm. Here is the list:\\
Input number            Algorithm type\\
    1                    Next fit\\
    2                    Random fit\\
    3                    Worst fit\\
    4                    Best fit\\
    0 or *               First fit\\

\section{Design details}

\subsection{Make}

\begin{verbatim}

setup.sh{
  #!/bin/sh
  cd /usr/src/servers
  make image
  make install
  cd /usr/src
  make libraries
  cd /usr/src/tools
  make hdboot
  make install
}
\end{verbatim}


\subsection{System Call}

I started in /usr/src/servers/pm/ in which I modified table.c and proto.h. I added: \\
{\bf pm.h}
added int algo\_type\\
{\bf table.c}\\
\begin{verbatim}
added memory\_call to slot 69:
memory\_call,    /* 69 = egall mem call  */
\end{verbatim}


{\bf proto.h}\\
\begin{verbatim}
added the following to misc.c:
\_PROTOTYPE( int memory\_call, (void)                                     ); \\
\end{verbatim}

{\bf misc.c}
\begin{verbatim}
PUBLIC int memory_call(){
  printf("\I am a system call \n");
  return(OK);
}
\end{verbatim}


In /usr/src/include/minix.callnr.h I added: \\
\#define MEMORY            69 \\

In /usr/src/lib/posix I added the file memory.c with the contents\\

{\bf \_memcall.c}
\begin{verbatim}
PUBLIC int memcall(int algotype){
  message m;
  m.m1\_i1 = algotype;
  return(\_syscall(MM, MEMCALL, &m);
}
\end{verbatim}

\section{Memory allocation algorithms}

\subsection{First Fit}

{\bf Psudo code}

\begin{verbatim}
while(There are still holes in the list){
  if(Length of current hole is >= # of clicks){
    old_base = current_base;
    current_length -= # of clicks;
    current_base += # of clicks;
    if(current_base > high watermark){
      high_watermark = current_base;
    }
    if(current_length == 0){
      del_slot(previous, current)
    }
    return(old_base);
  }
  previous = current;
  current = next;
}
\end{verbatim}

{\bf Next fit}

\begin{verbatim}
while(There are still holes in the list: Start at curr_fit){
  if(Length of current hole is >= # of clicks){
    old_base = current_base;
    current_length -= # of clicks;
    current_base += # of clicks;
    if(current_base > high watermark){
      high_watermark = current_base;
    }
    if(current_length == 0){
      del_slot(previous, current)
    }
    return(old_base);
  }
  previous = current;
  current = next;
}
\end{verbatim}

{\bf Random fit}

\begin{verbatim}
while(0){
  rand = ramdom number;
  rand = (rand%#of holes)
  for(;current_hole != NULL; current++){
    if(length of curr hole is >= # of clicks){
      old_base = current_base;
      current_length -= # of clicks;
      current_base += # of clicks;
      if(current_base > high watermark){
        high_watermark = current_base;
      }
      if(current_length == 0){
        del_slot(previous, current);
      }
      return(old_base);
    }
  }
}
\end{verbatim}

{\bf Worst fit}

\begin{verbatim}
int max;
int slot;
for(i;all holes in list){
  if(max < length of curr hole && length of curr hole >= # of clicks){
    max = current_length;
    slot = i;
  }
}
for(i > slot; i++){
  curr = curr->next);
}
old_base = current_base;
current_length -= # of clicks;
current_base += # of clicks;
if(current_base > high watermark){
  high_watermark = current_base;
}
if(current_length == 0){
  del_slot(previous, current);
}
return(old_base);
\end{verbatim}


{\bf Best fit}


\begin{verbatim}
int min;
int slot;
for(i;all holes in list){
  if(min < length of curr hole && length of curr hole >= # of clicks){
    max = current_length;
    slot = i;
  }
}
for(i > slot; i++){
  curr = curr->next);
}
old_base = current_base;
current_length -= # of clicks;
current_base += # of clicks;
if(current_base > high watermark){
  high_watermark = current_base;
}
if(current_length == 0){
  del_slot(previous, current);
}
return(old_base);
\end{verbatim}


\section{Testing}
In order to test the program, copy all of the files over to their proper places. Run the setup.sh shell script. Shutdown, startup with the 3rd option. Run test.c.\\

\section{Conclusion}
I got a lot more practice with changing the operating system on a virtual machine. I feel a lot more comfortable working with Minix after writing this program out.\\
I learned how to impliment a system call in the Minix operating system. I feel confident in implimenting other system calls if need be. I also feel that I am prepared to impliment a system call in a distro of Linux or MacOS if I were so inclined.\\

\end{document}

